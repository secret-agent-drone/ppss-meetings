\documentclass[11pt, oneside]{article}   	% use "amsart" instead of "article" for AMSLaTeX format
\usepackage{geometry}                		% See geometry.pdf to learn the layout options. There are lots.
\geometry{letterpaper}                   		% ... or a4paper or a5paper or ... 
%\geometry{landscape}                		% Activate for for rotated page geometry
%\usepackage[parfill]{parskip}    		% Activate to begin paragraphs with an empty line rather than an indent
\usepackage{graphicx}				% Use pdf, png, jpg, or eps§ with pdflatex; use eps in DVI mode
								% TeX will automatically convert eps --> pdf in pdflatex		
\usepackage{amssymb}

\title{Academic Programs Committee notes}
\author{Innes Bigaran }
\date{11th March 2016}							% Activate to display a given date or no date

\begin{document}
\maketitle
\section{Report on new level 2 subjects, management of legacy subjects and subject advice}
\subsection{First year}
There have been some issues with advanced physics and students being allowed to enrol in the subjects without the prerequisites. They will be informed about being withdrawn from the subject- not as a reflection of their competence but rather so that people are NOT further advised at high school level that certain prerequisites are not required- hence there isn't any sort of additional decrease in people taking physics to a senior level in VCE. \\

As per last year, there is hope that towards the later parts of semester there will be additional tutorials run in the lab centre on specific topics such as Differential Equations and Gauss' Law...

\subsection{ISIS/Handbook}
There has been an issue with ISIS and crossing over the old and new second year subjects- ISIS wasn't updated consistently with the handbook and hence prerequisites and such were not consistent. There has also been issues with overfilling of subjects- and students are being advised to sit in on the lectures/tutorials they can go to.

\subsection{Second year}
Approximately 200 students enrolled- in legacy and new subjects. There are issues where students have needed to take legacy subjects but are being credited for taking the new subject equivalent/s- there were issues with the access of the LMS for people enrolled in legacy but needing to access the LMS for the new subjects, but Sean Crosby is a genius and reprogrammed the LMS so that this issue didn't exist. \\

The new second year lab classes should increase the number of non physics students taking physics subjects- especially engineering, as they may want more practical learning. Therefore in the future we will need to prepare for larger class numbers- but we are overfilled/ at capacity now. So far they have booked the Thomas Cherry labs for computational labs, but are working on getting access to the Graduate Science computer labs in Laby South- the Laby centre is "reserved" for astro labs. \\

\section{Proposed pre-requisite change to PHYC20013: Laboratory and Computational Physics 2}

Previously the pre-requisites are Physics 1, Physics 2, Calculus 2 and Linear algebra. The argument is that one would only really require Physics 1 and Physics 2 to be able to do the subject- this is especially good for students who did UMEP physics and can't enrol in any physics, consequentially, in first year due to bureaucracy. They will still need to have taken 50 points of level 1 before they enrol in level 2 subject, but with this proposed change then UMEP students will be able to enrol in Lab and comp from second semester of this year. The change was voted on and passed. 

\section{Change to Major in Physics: Laboratory and Computational Physics 3}
Now that there is a Lab and Comp second year subject, we have to create and Lab and Comp third year subject for next year. The requirement for the Major in physics will now be: Compulsory- QM and Lab and Comp, One of- Electrodynamics or Statistical, One of the options- as per previously. 

\section{Teaching Support Fellows}
Have employed 5 teaching support fellows- "Senior tutors/demonstrators". So far they are five males- there were 6 positions but one is being held open for the sake of gender diversity. They will advertise again for these roles in second semester, and are considering advertising one as a "Female teaching support fellow" for the sake of diversity- affirmative action. 

\section{Experimental and Computational Laboratories report}
There has been really good feedback for second year labs- especially for the computational projects. 

\section{SSLC and PPSS}
SSLC meeting is 22nd March. Welcome to Innes, the new PPSS APC representative. 





\end{document}  